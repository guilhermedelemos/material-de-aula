%%%%%%%%%%%%%%%%%%%%%%%%%%%%%%%%%%%%%%%%%
% REDES DE COMPUTADORES:
% Oficina de servidores HTTP.
%
% Autor:  GUILHERME DE LEMOS
% E-mail: guilherme.eti@gmail.com
% GitHub: https://github.com/guilhermedelemos
% Data:   2018-06-13 02:09 UTC
%%%%%%%%%%%%%%%%%%%%%%%%%%%%%%%%%%%%%%%%%
% LaTeX Template: see README.md
%%%%%%%%%%%%%%%%%%%%%%%%%%%%%%%%%%%%%%%%%
\documentclass{beamer}

%----------------------------------------------------------------------------------------
%	PACKAGES AND THEMES
%----------------------------------------------------------------------------------------
\mode<presentation> {

\usetheme{Madrid}

%\setbeamertemplate{footline} % To remove the footer line in all slides uncomment this line

%\setbeamertemplate{footline}[page number] % To replace the footer line in all slides with a simple slide count uncomment this line

%\setbeamertemplate{navigation symbols}{} % To remove the navigation symbols from the bottom of all slides uncomment this line
}

\usepackage{graphicx} % Allows including images
\usepackage{booktabs} % Allows the use of \toprule, \midrule and \bottomrule in tables
\usepackage[utf8]{inputenc}
\usepackage[portuguese]{babel}
\usepackage[T1]{fontenc}

%----------------------------------------------------------------------------------------
%	TITLE PAGE
%----------------------------------------------------------------------------------------

\title[Oficina: Servidores HTTP]{Redes de Computadores: Oficina de Servidores HTTP}

\author{Guilherme de Lemos} % Your name
\institute[REDES] % Your institution as it will appear on the bottom of every slide, may be shorthand to save space
{
https://github.com/guilhermedelemos \\ % Your institution for the title page
\medskip
\textit{guilherme.eti@gmail.com}
}
\date{\today} % Date, can be changed to a custom date

\begin{document}

\begin{frame}
\titlepage % Print the title page as the first slide
\end{frame}

\begin{frame}
\frametitle{Overview} % Table of contents slide, comment this block out to remove it
\tableofcontents % Throughout your presentation, if you choose to use \section{} and \subsection{} commands, these will automatically be printed on this slide as an overview of your presentation
\end{frame}

%----------------------------------------------------------------------------------------
%	PRESENTATION SLIDES
%----------------------------------------------------------------------------------------

%------------------------------------------------
\section{Introdução}
\begin{frame}
\frametitle{Introdução}
Introdução da apresentação
\end{frame}

%------------------------------------------------
\subsection{Bibliografia Básica}
\begin{frame}
\frametitle{Bibliografia básica}
Mínimo necessário a ser lido.
\end{frame}

%------------------------------------------------
\subsection{Objetivos}
\begin{frame}
\frametitle{Objetivos}
Objetivos da oficina
\end{frame}

%------------------------------------------------
\subsection{Conceitos de base}
\begin{frame}
\frametitle{Conceitos de Base}
O mínimo necessário para fazer a oficina.

\begin{block}{Segundo \cite{p1} servidores HTTP são:}
Lorem ipsum dolor sit amet, consectetur adipiscing elit. Integer lectus nisl, ultricies in feugiat rutrum, porttitor sit amet augue. Aliquam ut tortor mauris. Sed volutpat ante purus, quis accumsan dolor.
\end{block}

\end{frame}

%------------------------------------------------
\section{Servidores HTTP}
\begin{frame}
\frametitle{Servidores HTTP}
Nesta oficina veremos dois dos principais servidores HTTP mais usados no mercado.

\begin{table}
\begin{tabular}{l l l}
\toprule
- & \textbf{Apache2} & \textbf{Nginx}\\
\midrule
\textbf{Lançamento} & 1990? & 2008? \\
\textbf{Usuários}   & 3BI? & 2BI \\
\textbf{Licença}    & Apache? & GPL? \\
\bottomrule
\end{tabular}
\caption{Servidores abordados}
\end{table}
\end{frame}

%------------------------------------------------
\subsection{Apache2}
\begin{frame}
\frametitle{Servidor HTTP Apache2}
Servidor HTTP Apache2
\end{frame}

%------------------------------------------------
\subsubsection{Instalação}
\begin{frame}
\frametitle{Instalação}

Inst

\end{frame}

%------------------------------------------------
\subsection{Nginx}
\begin{frame}
\frametitle{Servidor HTTP NGinx}
Servidor HTTP Nginx
\end{frame}

%------------------------------------------------
\section{Conclusões}
\begin{frame}
\frametitle{Conclusões}

\begin{itemize}
\item Lorem ipsum dolor sit amet, consectetur adipiscing elit
\item Aliquam blandit faucibus nisi, sit amet dapibus enim tempus eu
\item Nulla commodo, erat quis gravida posuere, elit lacus lobortis est, quis porttitor odio mauris at libero
\item Nam cursus est eget velit posuere pellentesque
\item Vestibulum faucibus velit a augue condimentum quis convallis nulla gravida
\end{itemize}

\end{frame}

%------------------------------------------------
\section{Referências}
\begin{frame}
\frametitle{Referências}
\footnotesize{
\begin{thebibliography}{99} % Beamer does not support BibTeX so references must be inserted manually as below
\bibitem[Smith, 2012]{p1} John Smith (2012)
\newblock Title of the publication
\newblock \emph{Journal Name} 12(3), 45 -- 678.
\end{thebibliography}
}
\end{frame}

%------------------------------------------------
\begin{frame}
\frametitle{Agradecimentos}
\begin{center}
{\huge \textbf{Obrigado!}}\\~\\

Dúvidas, sugestões, críticas, etc. envie e-mail para\\
guilherme.eti@gmail.com\\~\\

Este material é disponibilizado sob a licença Creative Commons.\\~\\

Fork-me on GitHub.\\
https://github.com/guilhermedelemos/material-de-aula\\~\\
\end{center}

\end{frame}


\end{document}
